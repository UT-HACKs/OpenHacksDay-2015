\documentclass[a4j,12pt,twoside]{tetsujsarticle}
\usepackage[top=30truemm,bottom=30truemm,left=25truemm,right=25truemm]{geometry}
\usepackage{tetsuryoku}
\usepackage{booktabs}
\begin{document}
\title{実験レポート\\Open Hack Day 3}
\author{UT-HACKs}
\maketitle\begin{center}
\img<0.45>{application.macosx/fig/graph.png}\\
\begin{tabular}{rrrrr}\toprule 
No.&l&m&t&v\\
\midrule
1&82&1073&41&49\\
2&76&410&155&40\\
3&16&146&47&39\\
4&76&1131&29&28\\
5&25&168&118&62\\
6&1334&1516&211&41\\
7&60&1044&127&35\\
8&72&652&299&31\\
9&43&1073&89&18\\
10&75&1158&59&76\\
\bottomrule
\end{tabular}
\end{center}
\section{特色}
 Camiappで取り込んだデータが10秒で \TeX の表とグラフに!\end{document}